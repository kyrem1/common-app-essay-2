\documentclass[12pt]{article}

%%%%%%%%%%%%%%%%%%%%%%%%%%%%%%%%%%-- Settings --%%%%%%%%%%%%%%%%%%%%%%%%%%%%%%%%%%%%%%%%%%%
\usepackage[english]{babel}

% - Margin - 1 inch on all sides
\usepackage[letterpaper]{geometry}
\usepackage[utf8]{inputenc}
\geometry{top=1.0in, bottom=1.0in, left=1.0in, right=1.0in}

% - Double Spacing -
\usepackage{setspace}
\doublespacing
\setstretch{2.00}

% Indent First paragraph
\usepackage{indentfirst}

% Other Packages
\usepackage{outlines}

% Title Settings
\title{
    \vspace{2in}
    \textmd{\textbf{Common Application - Essay Portion - 2nd Attempt}}\\
    \normalsize\vspace{0.1in}\small{Due\ Date\ Unknown}\\
    \vspace{3in}
}
\author{James Harbour}

\renewcommand{\footnoterule}{%
  \kern -3pt
  \hrule width \textwidth height 0.5pt
  \kern 2pt
}

%%%%%%%%%%%%%%%%%%%%%%%%%%%%%%%%%%-- Assignment --%%%%%%%%%%%%%%%%%%%%%%%%%%%%%%%%%%%%%%%%%%%

  % Common Application : general essay
  % Words Min: 250
  % Words Max: 650

%%%%%%%%%%%%%%%%%%%%%%%%%%%%%%%%%%-- TODOLIST --%%%%%%%%%%%%%%%%%%%%%%%%%%%%%%%%%%%%%%%%%%%

  % COMP Write tonesetting intro/hook to be placed before hook 2.
  % TODO Link Hook 1 to Hook 2
  % TODO Shave technicality
  % TODO Intersperse flashbacks to critical moments in my life
  % TODO Add being a member of Pi Mu Epsilon as a high schooler at the end.
  % TODO Shave verbosity
  % TODO Resolve meaning of Q.E.D. from the inside at the end of the essay

%%%%%%%%%%%%%%%%%%%%%%%%%%%%%%%%%%-- Document --%%%%%%%%%%%%%%%%%%%%%%%%%%%%%%%%%%%%%%%%%%%

% Begin Document
\begin{document}
\maketitle
\pagebreak
 \begin{center}

   Title 1: \emph{Q.E.D.} \\
   Title 2: \emph{Gazing into the Depths}

 \end{center}
\raggedright\setlength{\parindent}{0.5in}

%%% Intro / Hook 1 %%%

% TODO Set tone of essay with intro

Inspiration appears in many colors. Mine arrived in black and white accompanied by the phrase \emph{quod erat demonstrandum}, Latin for ``which was to be demonstrated.'' Mathematicians follow a tradition of appending this phrase to the end of a proof to signify its completion. From the outside, its uniformity throughout the subject appears to be an eccentricity of crazy people. I used to agree, but now I am no longer on the outside. % N.B. Resolve this at the end of the essay

% TODO Link Hook 1 to Hook 2

%%% Hook 2 %%%
Upon finally understanding the ingenious proof of Cantor’s theorem, I once again felt that familiar sense of wonder towards the giants whose shoulders I now aspire to stand upon. Following this moment of reverence, I experience the ineffable pleasure of rightly viewing mathematical beauty.

%%% Main Body %%%

After furiously writing the next line of the proof in my notes, I stopped. My professor continued his exposition, but I did not continue writing. Just then, my mind entered into a familiar state characterized by a refusal to acknowledge the passage of time.

  %%% Flashback to 7th grade alg 1 midterm as first time I experienced this state %%%

  The first time I had ever experienced this state of mind was during the winter break of 7th grade, before I had ever heard of the words \emph{quod erat demonstrandum}. I had convinced the head of the math department to allow to move from pre-algebra to algebra 1 after the break so that I could experience the academic challenge that I so desperately needed. However, on the last day before winter break I learned that his acceptance was contingent upon one seemingly impossible condition: that I score near perfect on the algebra 1 midterm exam, having only two weeks to prepare.

  Despite experiencing countless struggles throughout those next two weeks, at the end I gained a new superpower: the ability to truly focus. I could never have forseen how much passing that exam would change the trajectory of my life for the better.

  % TODO link above flashback to next

The line did not make sense; however, that is not to say that I did not understand what the line proposed, quite the opposite, in fact. Omitting the finer details, the line involved the construction of a set with peculiar, but completely unmotivated conditions placed upon its elements. Due to this lack of motivation, I simply could not fathom the object's place in the overall proof. % TODO Parallelism to how my the object's place in the proof mirrors my place in society, i.e. an outsider or something edgy like that

The argument appeared to be a standard proof by contradiction, where arriving at an absurdity after only assuming the falsity of a statement necessarily implies the truth of said statement. Yet the construction of this object threw any familiarity with this proof method out of the window.

But then I remembered that it was precisely this argument that began a revolution in the field of mathematics as well as philosophy. It was precisely this argument that solidified the immortality of its creator, Georg Cantor.

Realizing that such a revolutionary proof must assuredly contain fundamentally new and foreign ideas, I accepted my lack of understanding and continued following along with my professor.

As the proof neared the end, the argument finally landed upon the desired absurdity: the existence of an object which is simultaneously in and not in the previously constructed set, an obvious contradiction.

Following the argument's conclusion, the shuffling of college students towards the exit signaled the end of the lecture, but I stubbornly continued to stare at my notes. My mind refused to quit until I was certain that I understood how the strange object's appearance led to such an obvious absurdity. Scanning over each line, searching for an inkling of insight, I trudged onward. Likely thanks to Einstein, any awareness of the passage of time flew out the nonexistent window of the basement classroom. By the time I had finally grasped the idea of the proof, the room was quite barren. Noticing me looking up, my professor simply asked me if I understood. Nodding, I simply responded, "Wow."

When gazing upon the development of scientific thought, I used to only see unreachable giants as the impetus for dramatic changes. In that moment, when I emerged victorious against the gibberish in my notes, I sat upon the shoulders of a giant and realized that he was just a mortal, but his idea was immortal. It is immortal ideas such as these that ultimately go on to dramatically change history, and, in doing so, grant their immortality to the mortals who conceived of them.

Thinking back upon this experience over and over, I eventually ascertained that I too desire to create something immortal.

The above experience



% TODO Resolve meaning of Q.E.D. from the inside **at the end of the essay**
%   - Now I understand that, from the inside, Q.E.D. is the final note at the end of the
%     concerto. (or something like this)



\end{document}
